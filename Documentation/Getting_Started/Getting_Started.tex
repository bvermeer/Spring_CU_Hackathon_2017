%%%%%%%%%%%%%%%%%%%%%%%%%%%%%%%%%%%%%%%%%
% University/School Laboratory Report
% LaTeX Template
% Version 3.0 (4/2/13)
%
% This template has been downloaded from:
% http://www.LaTeXTemplates.com
%
% Original author:
% Linux and Unix Users Group at Virginia Tech Wiki 
% (https://vtluug.org/wiki/Example_LaTeX_chem_lab_report)
%
% License:
% CC BY-NC-SA 3.0 (http://creativecommons.org/licenses/by-nc-sa/3.0/)
%
%%%%%%%%%%%%%%%%%%%%%%%%%%%%%%%%%%%%%%%%%

%----------------------------------------------------------------------------------------
%	PACKAGES AND DOCUMENT CONFIGURATIONS
%----------------------------------------------------------------------------------------

\documentclass{article}

\usepackage[version=3]{mhchem} % Package for chemical equation typesetting
\usepackage{siunitx} % Provides the \SI{}{} command for typesetting SI units

\usepackage[top=1in, bottom=1in, right=1in, left=1in]{geometry}

%Add code formating
\usepackage{listings}
\lstset{tabsize=2}

\usepackage{hyperref}

\usepackage{amssymb}

\usepackage{enumerate}

%Add extra support for image placement
\usepackage{float}

\usepackage{mcode}

\usepackage{graphicx} % Required for the inclusion of images

\setlength\parindent{0pt} % Removes all indentation from paragraphs

\renewcommand{\labelenumi}{\alph{enumi}.} % Make numbering in the enumerate environment by letter rather than number (e.g. section 6)

%\usepackage{times} % Uncomment to use the Times New Roman font

%----------------------------------------------------------------------------------------
%	DOCUMENT INFORMATION
%----------------------------------------------------------------------------------------

\title{Keysight Hackathon Dev Kit Getting Started} % Title

\author{Blake \textsc{Vermeer}} % Author name

\date{\today} % Date for the report

\begin{document}

\maketitle % Insert the title, author and date

\begin{center}
\begin{tabular}{l r}
Date Performed: & March 21, 2017 \\ % Date the experiment was performed
Company: & Keysight Technologies % Company
\end{tabular}
\end{center}

% If you wish to include an abstract, uncomment the lines below
% \begin{abstract}
% Abstract text
% \end{abstract}

%----------------------------------------------------------------------------------------
%	OVERVIEW
%----------------------------------------------------------------------------------------
\section{Overview}

The Keysight Hackathon Dev Kit consists of:

	\begin{itemize}
		
		\item Raspberry Pi 3 with 2.8" capacitive touch screen shield
		
		\item Custom Yocto Linux image pre-loaded on the Raspberry Pi 3
		
		\item Fedora Linux Virtual Machine image with Qt Creator and the Yocto SDK pre-installed and configured with Qt Creator
		
	\end{itemize}



%----------------------------------------------------------------------------------------
%	Installing VirtualBox
%----------------------------------------------------------------------------------------
\section{Installing VirtualBox}




%----------------------------------------------------------------------------------------
%	Importing the Virtual Machine
%----------------------------------------------------------------------------------------
\section{Import the Virtual Machine}




%----------------------------------------------------------------------------------------
%	Launching the Virtual Machine
%----------------------------------------------------------------------------------------
\section{Launching the Virtual Machine}




%----------------------------------------------------------------------------------------
%	Start Qt Creator and Connect to the Raspberry Pi
%----------------------------------------------------------------------------------------
\section{Start Qt Creator and Connect to the Raspberry Pi}




%----------------------------------------------------------------------------------------
%	APPENDIX
%----------------------------------------------------------------------------------------

%\newpage
%\section{Appendix}

%\begin{enumerate}

	
%	\item[1. a.)] \lstinputlisting{../MATLAB/problem_1a.m}
	

%\end{enumerate}






%----------------------------------------------------------------------------------------


\end{document}