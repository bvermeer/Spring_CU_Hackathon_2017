%%%%%%%%%%%%%%%%%%%%%%%%%%%%%%%%%%%%%%%%%
% University/School Laboratory Report
% LaTeX Template
% Version 3.0 (4/2/13)
%
% This template has been downloaded from:
% http://www.LaTeXTemplates.com
%
% Original author:
% Linux and Unix Users Group at Virginia Tech Wiki 
% (https://vtluug.org/wiki/Example_LaTeX_chem_lab_report)
%
% License:
% CC BY-NC-SA 3.0 (http://creativecommons.org/licenses/by-nc-sa/3.0/)
%
%%%%%%%%%%%%%%%%%%%%%%%%%%%%%%%%%%%%%%%%%

%----------------------------------------------------------------------------------------
%	PACKAGES AND DOCUMENT CONFIGURATIONS
%----------------------------------------------------------------------------------------

\documentclass{article}

\usepackage[version=3]{mhchem} % Package for chemical equation typesetting
\usepackage{siunitx} % Provides the \SI{}{} command for typesetting SI units

\usepackage[top=1in, bottom=1in, right=1in, left=1in]{geometry}

%Add code formating
\usepackage{listings}
\lstset{tabsize=2}

\usepackage{hyperref}

\usepackage{amssymb}

\usepackage{enumerate}

\usepackage{multicol} % Multi-column support

%Add extra support for image placement
\usepackage{float}

\usepackage{mcode}

\usepackage{graphicx} % Required for the inclusion of images

\setlength\parindent{0pt} % Removes all indentation from paragraphs

\renewcommand{\labelenumi}{\alph{enumi}.} % Make numbering in the enumerate environment by letter rather than number (e.g. section 6)

%\usepackage{times} % Uncomment to use the Times New Roman font

%----------------------------------------------------------------------------------------
%	DOCUMENT INFORMATION
%----------------------------------------------------------------------------------------

\title{Keysight Hacking Platform Hardware Overview} % Title

\author{Blake \textsc{Vermeer}} % Author name

\date{\today} % Date for the report

\begin{document}

\maketitle % Insert the title, author and date

\begin{center}
\begin{tabular}{l r}
Date Performed: & March 26, 2017 \\ % Date the experiment was performed
Company: & Keysight Technologies % Company
\end{tabular}
\end{center}

% If you wish to include an abstract, uncomment the lines below
% \begin{abstract}
% Abstract text
% \end{abstract}

%----------------------------------------------------------------------------------------
%	OVERVIEW
%----------------------------------------------------------------------------------------
\section{Overview}

This document gives a general hardware architecture overview of the Keysight Hacking Platform. A general hardware overview of the Raspberry Pi 3 and then a detailed description of how the touch-screen is connected to the Raspberry Pi 3 is given in the sections below.


%----------------------------------------------------------------------------------------
%	RPi 3 Block Diagram
%----------------------------------------------------------------------------------------
\section{Raspberry Pi 3 Block Diagram}

At the heart of the Keysight Hacking Platform is a Raspberry Pi 3. The Raspberry Pi 3 is a general purpose embedded ARM Linux device. 

	\begin{minipage}{0.5\textwidth}
		
		\includegraphics[width=0.95\textwidth]{pics/Raspberry-Pi-3_Block_Diagram.jpg}
		
		%\begin{figure}[H]
		%	\centering
		%	\includegraphics[scale=0.3]{pics/Raspberry-Pi-3_Block_Diagram.jpg}
		%	\caption{Raspberry Pi 3 Block Diagram}
		%	\label{RPi_3_Block_Diagram}
		%\end{figure} 
	
	\end{minipage}
	\begin{minipage}{0.45\textwidth}
		
		\begin{itemize}
			\item CPU: 1.2 GHz quad-core ARM Cortex A53
			\item Memory: 1 GB LPDDR2-900 SDRAM
			\item 4 USB ports (Max current draw of 1.2A combined on all the USB ports)
			\item 10/100 Ethernet
			\item HDMI
			\item Bluetooth 4.0
			\item 802.11n Wireless LAN
			\item Combination RCA Video / Audio jack
			\item 40 Pin GPIO Connector
		\end{itemize}
	
	\end{minipage}








%----------------------------------------------------------------------------------------
%	RPi 3 GPIO Usage Overview
%----------------------------------------------------------------------------------------
\section{Raspberry Pi 3 Used GPIO Lines}

This section explains which GPIO lines are used by the touchscreen and which GPIO lines are connected to the four push-buttons. The LCD is driven by an SPI interface. The Raspberry Pi 3 contains three independent SPI bus drivers and in the case the screen is connect to SPI bus 0. The Raspberry Pi 3 has two I2C buses available on the GPIO header and both are used by the screen (I2C bus 0 is used by the configuration EEPROM and I2C bus 1 is used by the touchscreen). Figure \ref{Screen_Used_GPIO} shows the GPIO lines used by the touchscreen and the LCD.

	\begin{figure}[H]
		\centering
		\includegraphics[scale=0.3]{pics/GPIO_Used_By_Screen.png}
		\caption{GPIO Lines Used by the Screen}
		\label{Screen_Used_GPIO}
	\end{figure} 

	\begin{table}
		\centering
		
		\begin{tabular}[H]{| c | c |}
			\hline
			\textbf{Communication Bus} & \textbf{Used by} \\
			\hline
			I2C Bus 0 & Configuration EEPROM \\
			\hline
			I2C Bus 1 & Touchscreen \\
			\hline
			SPI Bus 0 & LCD \\
			\hline
			SPI Bus 1 & nothing \\
			\hline
			SPI Bus 2 & nothing \\
			\hline
		\end{tabular}
	\end{table}


The touch-screen also features four hardware push-buttons which are connected directly to GPIO lines are shown in the section of the touch-screen schematic shown in Figure \ref{Push_buttons_schematic}.


	\begin{figure}[H]
		\centering
		\includegraphics[width=0.25\textwidth]{pics/PiTFT_2-8_push-buttons_section_schematic.png}
		\caption{Screen Push-Buttons Schematic}
		\label{Push_buttons_schematic}
	\end{figure}


In order to use the push-buttons, the GPIO lines connected to the push-buttons need to be configured as inputs and the internal pull-up resistors enabled (external pull-up resistors could alternatively be used). The push-buttons are connected to the GPIO lines on the Raspberry Pi 3 header as shown in Figure \ref{Push_buttons_GPIO}.


	\begin{figure}[H]
		\centering
		\includegraphics[scale=0.3]{pics/PiTFT_2-8_Switches_GPIO.png}
		\caption{GPIO Lines used by Push-Buttons}
		\label{Push_buttons_GPIO}
	\end{figure}


It is also possible to figure out which push-button is connected to which GPIO line by looking at the silkscreen labels next to the push-buttons on the screen. Note: the numbers on the labels are the Broadcom pin numbers (BCM numbers) and not the physical pin numbers on the header!



	\begin{figure}[H]
		\centering
		\includegraphics[width=0.65\textwidth]{pics/PiTFT_Plus_2_8_Overhead.jpg}
		\caption{Push Button Labels}
		\label{Push_Button_Labels}
	\end{figure}


In summary, Figure \ref{All_Used_GPIO} shows all the GPIO lines in use.


	\begin{figure}[H]
		\centering
		\includegraphics[scale=0.3]{pics/All_Used_GPIO.png}
		\caption{All Used GPIO}
		\label{All_Used_GPIO}
	\end{figure}

%----------------------------------------------------------------------------------------
%	RPi 3 Available GPIO and Interfaces
%----------------------------------------------------------------------------------------
\section{Raspberry Pi 3 Available GPIO and Interfaces}




%----------------------------------------------------------------------------------------
%	APPENDIX
%----------------------------------------------------------------------------------------

%\newpage
%\section{Appendix}

%\begin{enumerate}

	
%	\item[1. a.)] \lstinputlisting{../MATLAB/problem_1a.m}
	

%\end{enumerate}






%----------------------------------------------------------------------------------------


\end{document}